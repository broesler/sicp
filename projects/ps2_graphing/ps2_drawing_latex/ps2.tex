%\documentstyle[6001]{article}
% Document Type: LaTeX
% Master File: ps3rsa.tex

%Warning: This was Latex'ed on a Mac in order to include the figures
% this next is to generate figures on the Mac

%\def\picture here

\def\picture #1 by #2 (#3){
  $${\vbox to #2{
    \hrule width #1 height 0pt depth 0pt
    \vfill
    \special{picture #3} % this is the low-level interface
    }}$$
  }
\input ./6001mac.tex
%\input /zu/u6001/6001mac
%\input /b/meyer/6001/94-dir/pub-dir-F94/6001mac

\let\to\rightarrow
\let\union\cup
\let\cross\times
\def\SN{\mbox{Sch-\-Num}}
\def\SB{\mbox{Sch-Bool}}
\def\Empty{\mbox{Empty}}
\def\SI{\mbox{Sch-Int}}
\def\SNI{\mbox{Sch-Nonneg-Int}}
\def\CRV{\mbox{Curve}}
\def\UI{\mbox{Unit-Interval}}
\def\PT{\mbox{Point}}
\def\UT{\mbox{Curve-Transform}}
\def\BT{\mbox{Binary-Transform}}

\def\fbox#1{%
  \vtop{\vbox{\hrule%
     \hbox{\vrule\kern3pt%
 \vtop{\vbox{\kern3pt#1}\kern3pt}%
 \kern3pt\vrule}}% 
 \hrule}}



\begin{document} 

%%%%%%%%%%%%%%%%%%%%%%%%%%%%%%%%%%%%%%%%%%%%%%%%%%%%%%%%%%%%%%%%%%%%%%%%%%%%%%%

\psetheader{SAMPLE} {Graphing Problem Set}

%%%%%%%%%%%%%%%%%%%%%%%%%%%%%%%%%%%%%%%%%%%%%%%%%%%%%%%%%%%%%%%%%%%%%%%%%%%%%%%

\begin{center}\large
{\bf Graphing with Higher-order Procedures}
\end{center}

\medskip

One of the things that makes Scheme different from other common
programming languages is the ability to operate with {\em higher-order
procedures}, namely, procedures that manipulate and generate other
procedures.  This problem set will give you extensive practice with
higher-order procedures, in the context of a language for graphing
two-dimensional curves and other shapes.  Sections 1 and 2 give some
background on the essential ideas, with exercises included.
Section 3, the actual programming assignment, describes the graphics
language and gives applications to generating fractal designs.  There
is also an optional design problem.


\section{1. Procedure Types and Procedure Constructors}

In this assignment we use many procedures which may be applied to many
different types of arguments and may return different types of values.  To
keep track of this, it will be helpful to have some simple notation to
describe types of Scheme values.

Two basic types of values are \SN, the Scheme numbers such as 3,
-4.2, 6.931479453e89, and \SB, the truth values {\tt \#t,\#f}.
The procedure {\tt square} may be applied to a \SN\ and will return
another \SN.  We indicate this with the notation:
\[{\tt square} : \SN \to \SN \]

If {\tt f} and {\tt g}  are procedures of type $\SN \to \SN$, then we may
{\em compose} them:

\beginlisp
(define (compose f g)
  (lambda (x)
    (f (g x))))
\endlisp

\noindent
Thus, for example {\tt (compose square log)} is the procedure of type $\SN
\to \SN$ that returns the square of the logarithm of its argument, while
{\tt (compose log square)} returns the logarithm of the square of its
argument:

\beginlisp
(log 2)
;Value: .6931471805599453
\null
((compose square log) 2)
;Value: .4804530139182014
\null
((compose log square) 2)
;Value: 1.3862943611198906
\endlisp

\noindent
As we have used it above, the procedure {\tt compose} takes as arguments
two procedures of type $F = \SN \to \SN$, and returns another such
procedure.  We indicate this with the notation:
\[{\tt compose} : (F, F) \to F\]

Just as squaring a number multiplies the number by itself, {\tt thrice} of
a function composes the function three times.  That is, {\tt ((thrice f)
n)} will return the same number as {\tt (f(f(f n)))}:

\beginlisp
(define (thrice f)
  (compose (compose f f) f))
\null
((thrice square) 3)
;Value: 6561
\null
(square (square (square 3)))
;Value: 6561
\endlisp

As used above, {\tt thrice} is of type $(F \to F)$.  That is, it takes as
input a function from numbers to numbers and returns the same kind of
function.  But {\tt thrice} will actually work for other kinds of input
functions.  It is enough for the input function to have a type of the form
$T\to T$, where $T$ may be any type.  So more generally, we can write
\[\hbox{\tt thrice} : (T\to T) \to (T\to T) \]

Composition, like multiplication, may be iterated.  Consider 
the following:

\beginlisp
(define (identity x) x)
\null
(define (repeated f n)
  (if (= n 0)
      identity
      (compose f (repeated f (- n 1)))))

\null

((repeated sin 5) 3.1)
;Value: 4.1532801333692235e-2

\null

(sin(sin(sin(sin(sin 3.1)))))
;Value: 4.1532801333692235e-2
\endlisp

\[\hbox{\tt repeated} : ((T\to T),\SNI) \to (T\to T)\]

\paragraph{Exercise 1.A} The type of {\tt thrice} is of the form
$(T'\to T')$ (where $T'$ happens to equal $(T\to T)$), so we can
legitimately use {\tt thrice} as an input to {\tt thrice}!

For what value of {\tt n} will {\tt (((thrice thrice) f) 0)} return the
same value\footnote{``Sameness'' of procedure values is a sticky issue
which we don't want to get into here.  We can avoid it by assuming that
{\tt f} is bound to a value of type $F$, so evaluation of {\tt (((thrice
thrice) f) 0)} will return a number.} as {\tt ((repeated f n) 0)}?

See if you can now predict what will happen when the following expressions
are evaluated.  Briefly explain what goes on in each case.
\begin{enumerate}
\item {\tt (((thrice thrice) 1+) 6)}
\item {\tt (((thrice thrice) identity) compose)}
\item {\tt (((thrice thrice) square) 1)}
\item {\tt (((thrice thrice) square) 2)}.
\end{enumerate}

\paragraph{Exercise 1.B}  Test your predictions.  ({\bf Warning}:
Before you do this, make sure you understand how to {\em interrupt} a 
Scheme evaluation.)

\section{2. Curves as Procedures and Data}

We're going to develop a language for defining and drawing planar curves.
We'd like to plot points, construct graphs of functions, transform curves
by scaling and rotating, and so on.  One of the key ideas that we'll
stress throughout 6.001 is that a well-designed language has parts that
combine to make new parts that themselves can be combined.  This property
is called {\em closure}.

A planar curve in ``parametric form'' can be described mathematically as a
function from parameter values to points in the plane.  For example, we
could describe the {\em unit-circle} as the function taking $t$ to $(\cos
2 \pi t, \sin 2 \pi t)$ where $t$ ranges over the unit interval $[0,1]$.  In Scheme,
we let \UI\ be the type of Scheme-numbers between 0 and 1, and we
represent curves by procedures of Scheme type \CRV, where
\[\CRV = \UI\to \PT\]
and \PT\ is some representation of pairs of \SN's.

To work with \PT, we need a {\em constructor}, {\tt make-point}, which
constructs \PT's from \SN's, and {\em selectors}, {\tt x-of} and {\tt y-of},
for getting the $x$ and $y$ coordinates of a \PT.  We require only that
the constructors and selectors obey the rules
\begin{eqnarray*}
(\hbox{\tt x-of}\, (\hbox{\tt make-point}\, n\, m)) & = & n\\
(\hbox{\tt y-of}\, (\hbox{\tt make-point}\, n\, m)) & = & m
\end{eqnarray*}
for all \SN's $m,n$.  Here is one way to do this (we'll learn several
other, better ways, in two weeks.)

\beginlisp
(define (make-point x y)
  (lambda (bit)
    (if (zero? bit) x y)))
\null
(define (x-of point)
  (point 0))
\null
(define (y-of point)
  (point 1))
\endlisp

\begin{eqnarray*}
\hbox{\tt make-point} & : & (\SN,\SN) \to \PT,\\
\hbox{\tt x-of}, \hbox{\tt y-of} & : & \PT\to \SN.
\end{eqnarray*}

For example, we can define the \CRV\ {\tt unit-circle} and the \CRV\ {\tt
unit-line} (along the $x$-axis):

\beginlisp
(define (unit-circle t)
  (make-point (sin (* 2pi t))
              (cos (* 2pi t))))
\null
(define (unit-line-at y)
  (lambda (t) (make-point t y)))
\null
(define unit-line (unit-line-at 0))

\endlisp

\paragraph{Exercise 2:}

\begin{enumerate}
\item What is the type of {\tt unit-line-at}?

\item Define a procedure {\tt vertical-line} with two arguments, a point and a
length, and returns a vertical line of that length beginning at the point.

\item What is the type of {\tt vertical-line}?

\end{enumerate}
\medskip

In addition to the direct construction of \CRV's such as {\tt unit-circle}
or {\tt unit-line}, we can use elementary Cartesian geometry in designing
Scheme procedures which {\em operate} on \CRV's.  For example, the mapping
$(x,y)\longrightarrow (-y,x)$ rotates the plane by $\pi/2$, so

\beginlisp
(define (rotate-pi/2 curve)
  (lambda (t)
    (let ((ct (curve t)))
      (make-point
       (- (y-of ct))
       (x-of ct)))))
\endlisp
defines a procedure which takes a curve and transforms it into another,
rotated, curve.  The type of {\tt rotate-pi/2} is
\[\UT = \CRV \to \CRV.\]

\paragraph{Exercise 3:}
Write a definition of a \UT\ {\tt reflect-through-y-axis}, which turns a
curve into its mirror image.

\medskip

We have provided a variety of other procedure \UT's and procedures which
construct \UT's in the file {\tt curves.scm}.  For example,
\begin{itemize}

\item 
{\tt translate} returns a \UT\ which rigidly moves a curve given distances
along the $x$ and $y$ axes,

\item {\tt scale-x-y}
returns a \UT\ which stretches a curve along the $x$ and $y$ coordinates
by given scale factors, and

\item {\tt rotate-around-origin}
returns a \UT\ which rotates a curve by a given number of radians.

\end{itemize}
A convenient, if somewhat more complicated, \UT\ is {\tt
put-in-standard-position}.  We'll say a curve is in {\em standard
position} if its start and end points are the same as the unit-line,
namely it starts at the origin, $(0,0)$, and ends at the point $(1,0)$.
We can put any curve whose start and endpoints are not the same into
standard position by rigidly translating it so its starting point is at
the origin, then rotating it about the origin to put its endpoint on the
$x$ axis, then scaling it to put the endpoint at $(1,0)$:

\beginlisp
(define (put-in-standard-position curve)
  (let* ((start-point (curve 0))
         (curve-started-at-origin
          ((translate (- (x-of start-point))
                      (- (y-of start-point)))
           curve))
         (new-end-point (curve-started-at-origin 1))
         (theta (atan (y-of new-end-point) (x-of new-end-point)))
         (curve-ended-at-x-axis
          ((rotate-around-origin (- theta)) curve-started-at-origin))
         (end-point-on-x-axis (x-of (curve-ended-at-x-axis 1))))
    ((scale (/ 1 end-point-on-x-axis)) curve-ended-at-x-axis)))
\endlisp

It is useful to have operations which combine curves into new ones.  We
let \BT\ be the type of binary operations on curves, \[\BT =
(\CRV,\CRV)\to \CRV.\] The procedure {\tt connect-rigidly} is a simple
\BT.  Evaluation of {\tt (connect-rigidly curve1 curve2)} returns a curve
consisting of {\tt curve1} followed by {\tt curve2}; the starting point of
the curve returned by {\tt (connect-rigidly curve1 curve2)} is the same as
that of {\tt curve1} and the end point is the same as that of {\tt
curve2}.

\beginlisp
(define (connect-rigidly curve1 curve2)
  (lambda (t) 
    (if (< t (/ 1 2))
        (curve1 (* 2 t))
        (curve2 (- (* 2 t) 1)))))
\endlisp

\paragraph{Exercise 4:} There is another, possibly more
natural, way of connecting curves.  The curve returned by {\tt
(connect-ends curve1 curve2)} consists of a copy of {\tt curve1}
follwed by a copy of {\tt curve2} after it has been rigidly translated
so its starting point coincides with the end point of {\tt curve1}.

Write a definition of the \BT\ {\tt connect-ends}.

\section{3. Drawing Curves}

Use {\tt M-x load-problem-set} to load the code for problem set 2.  This
will create three graphics windows called {\tt g1}, {\tt g2} and {\tt g3}.
The window coordinates go from 0 to 1 in both $x$ and $y$ with $(0,0)$ at
the lower left.

A {\em drawing procedure} takes a curve argument and automagically
displays points on the curve in a window\footnote{{\em The Hacker's
Dictionary} (see the Jargon file) defines ``automagically'' as
``automatically, but in a way which, for some reason (typically because it
is too complicated, or too ugly, or perhaps even too trivial), the speaker
doesn't feel like explaining.''  In this case, we don't want to explain
the bletcherous (see Jargon) details of how points are plotted.}.  We've
provided several procedures that take a window (for example {\tt g1}) and
a number of points, and return a drawing procedure, namely,
\begin{itemize}
\item {\tt draw-points-on},
\item {\tt draw-connected},
\item {\tt draw-points-squeezed-to-window}, and
\item {\tt draw-connected-squeezed-to-window}.
\end{itemize}

\paragraph{Exercise 5:}
Apply {\tt (draw-connected g1 200)} to {\tt unit-circle}, and {\tt
(draw-connected g2 200)} to {\tt alternative-unit-circle}.  Can you see a
difference?  Now try using {\tt draw-points-on} instead of {\tt
draw-connected}.  Also try {\tt draw-points-squeezed-to-window}.  Print
out the resulting figures.

\subsection{Fractal Curves}

To show off the power of our drawing language, let's use it to explore
fractal curves.  Fractals have striking mathematical
properties.\footnote{A fractal curve is a ``curve'' which, if you
expand any small piece of it, you get something similar to the
original.  The Gosper curve, for example, is neither a true
1-dimensional curve, nor a 2-dimensional region of the plane, but
rather something in between.} Fractals have received a lot of
attention over the past few years, partly because they tend to arise
in the theory of nonlinear differential equations, but also because
they are pretty, and their finite approximations can be easily
generated with recursive computer programs.

For example, Bill Gosper\footnote{Bill Gosper is a mathematician now
living in California.  He was one of the original hackers who worked
for Marvin Minsky in the MIT Artificial Intelligence Laboratory during
the '60s.  He is perhaps best known for his work on the Conway Game of
Life---a set of rules for evolving cellular automata.  Gosper invented
the ``glider gun'', resolving Conway's question as to whether it is
possible to produce a finite pattern that evolves into an unlimited
number of live cells.  He used this result to prove that the Game of
Life is Turing universal, in that it can be used to simulate any other
computational process!} discovered that the infinite repetition of a
very simple process creates a rather beautiful image, now called the
{\em Gosper C Curve}.  At each step of this process there is an
approximation to the Gosper curve.  The next approximation is obtained
by adjoining two scaled copies of the current approximation, each
rotated by 45 degrees.

Figure~\ref{Gosper} shows the first few approximations to the Gosper
curve, where we stop after a certain number of levels: a level-0 curve
is simply a straight line; a level-1 curve consists of two level-0
curves; a level 2 curve consists of two level-1 curves, and so on.
The figure also illustrates a recursive strategy for making the next
level of approximation: a level-$n$ curve is made from two
level-$(n-1)$ curves, each scaled to be ${\sqrt 2}/2$ times the length
of the original curve.  One of the component curves is rotated by
$\pi/4$ (45 degrees) and the other is rotated by $-\pi/4$.  After each
piece is scaled and rotated, it must be translated so that the ending
point of the first piece is continuous with the starting point of the
second piece.

We assume that the approximation we are given to improve (named {\tt
curve} in the procedure) is in standard position.  By doing some
geometry, you can figure out that the second curve, after being scaled
and rotated, must be translated right by .5 and up by .5, so its beginning
coincides with the endpoint of the rotated, scaled first curve.  This leads
to the \UT\ {\tt gosperize}:

%\label{Gosper} here

\begin{figure}
\picture 4.50 in by 2.54 in (C-curve)
\caption{{\protect\footnotesize
Examples of the Gosper C curve at various levels, and 
the recursive transformation that produces each level from the previous
level.}}
\label{Gosper}
\end{figure}

\beginlisp
(define (gosperize curve)
  (let ((scaled-curve ((scale (/ (sqrt 2) 2)) curve)))
    (connect-rigidly ((rotate-around-origin (/ pi 4)) scaled-curve)
                     ((translate .5 .5)
                      ((rotate-around-origin (/ -pi 4)) scaled-curve)))))
\endlisp

Now we can generate approximations at any level to the Gosper curve by
repeatedly gosperizing the unit line,

\beginlisp
(define (gosper-curve level)
    ((repeated gosperize level) unit-line))
\endlisp

To look at the level {\tt level} gosper curve, evaluate {\tt
(show-connected-gosper level)}:

\beginlisp
(define (show-connected-gosper level)
  ((draw-connected g1 200)
   ((squeeze-rectangular-portion -.5 1.5 -.5 1.5)
    (gosper-curve level))))
\endlisp

\paragraph{Exercise 6.A} Define a procedure {\tt show-points-gosper}
such that evaluation of

\beginlisp
(show-points-gosper window level number-of-points initial-curve)
\endlisp

\noindent
will plot {\tt number-of-points} unconnected points of the level {\tt
level} gosper curve in {\tt window}, but starting the gosper-curve
approximation with an arbitrary {\tt initial-curve} rather than the unit
line.  For instance,

\beginlisp
(show-points-gosper g1 level 200 unit-line)
\endlisp

\noindent should display the same points as {\tt
(show-connected-gosper level)}, but without connecting them.  But you
should also be able to use your procedure with arbitrary curves.  (You can
find the description of procedure {\tt squeeze-rectangular-portion} in the
file {\tt curves.scm}; you don't need to understand it in detail to do
this exercise.)

\paragraph{Exercise 6.B}
Try gosperizing the arc of the unit circle running from 0 to $\pi$.  Find
some examples that produce interesting designs.  (You may also want to
change the scale in the plotting window and the density of points
plotted.)\footnote{One of the things you should notice is that, for larger
values of $n$, all of these curves look pretty much the same.  As with
many fractal curves, the shape of the Gosper curve is determined by the
Gosper process itself, rather than the particular shape we use as a
starting point.  In a sense that can be made mathematically precise, the
``infinite level'' Gosper curve is a fixed point of the Gosper process,
and repeated applications of the process will converge to this fixed
point.}

\medskip

The Gosper fractals we have been playing with have had the angle of
rotation fixed at 45 degrees. This angle need not be fixed.  It need not
even be the same for every step of the process.  Many interesting shapes
can be created by changing the angle from step to step.

We can define a procedure {\tt param-gosper} that generates Gosper curves
with changing angles.  {\tt Param-gosper} takes a level number (the number
of levels to repeat the process) and a second argument called {\tt
angle-at}. The procedure {\tt angle-at} should take one argument, the level
number, and return an angle (measured in radians) as its answer
\[\hbox{\tt angle-at} : \SNI \to \SN.\]
Procedure {\tt param-gosper} can use this to calculate the angle to be
used at each step of the recursion.

\beginlisp
(define (param-gosper level angle-at)
  (if (= level 0)
      unit-line
      ((param-gosperize (angle-at level))
       (param-gosper (- level 1) angle-at))))
\endlisp


The procedure {\tt param-gosperize} is almost like {\tt gosperize}, except
that it takes an another argument, the angle of rotation, and implements
the process shown in figure~\ref{param-geo}:
        
\beginlisp        
(define (param-gosperize theta)
  (lambda (curve)
    (let ((scale-factor (/ (/ 1 (cos theta)) 2)))
      (let ((scaled-curve ((scale scale-factor) curve)))
        (connect-rigidly ((rotate-around-origin theta) scaled-curve)
                         ((translate .5 (* (sin theta) scale-factor))
                          ((rotate-around-origin (- theta)) scaled-curve)))))))
\endlisp

% \label{param-geo} here
\begin{figure}
\picture 4.14 in by 1.69 in (Param-geo)
\caption{{\protect\footnotesize
A parameterized version of the Gosper process, where the
angle can vary.  Note that the endpoints of the transformed figure
should be the same as the endpoints of the original figure, and
the interior endpoints should match.}}
\label{param-geo}
\end{figure}


For example, the ordinary Gosper curve at level {\tt level} is returned by

\beginlisp
(param-gosper level (lambda (level) pi/4))
\endlisp

\paragraph{Exercise 7.A}  Designing {\tt param-gosperize} required
using some elementary trigonometry to figure out how to shift the pieces
around so that they fit together after scaling and rotating.  It's easier
to program if we let the computer figure out how to do the shifting.  Show
how to redefine {\tt param-gosperize} using the procedures {\tt
put-in-standard-position} and {\tt connect-ends} from Exercise 4 to handle
the trigonometry.  Your definition should be of the form

\beginlisp
(define (param-gosperize theta)
  (lambda (curve)
    (put-in-standard-position
      (connect-ends
        ...
        ...))))
\endlisp


\paragraph{Exercise 7.B}
Generate some parameterized Gosper curves where the angle changes with the
level $n$.  We suggest starting with $\pi/(n+2)$ and $\pi/(1.3^n)$.
Submit sample printouts with your problem solutions.

\paragraph{Exercise 7.C}
We now have three procedures to compute gosper curves: {\tt gosper-curve},
and {\tt param-gosper} with argument {\tt (lambda (level) (/ pi 4)))}
using the ``hand-crafted'' definition of {\tt param-gosperize} above, or
using your version of {\tt param-gosperize} in 7.A based on {\tt
put-in-standard-position}.  Compare the speed of these procedures for
computing selected points on the curve at a few levels.  Is there a speed
advantage for the more customized procedures?

The procedure {\tt show-time} will report the time in milliseconds
required to evaluate a procedure of no arguments (a ``thunk'').  For
example, evaluating

\beginlisp
(show-time (lambda () ((gosper-curve 10) .1))
\endlisp
will print out the time to compute the point at .1 on the level 10 gosper-curve.

\medskip

\paragraph{Exercise 8.A}
Ben Bitdiddle isn't entirely happy with the style of several of the Scheme
definitions on this problem set. In particular, he feels the code goes
overboard in inventing names for values that are used infrequently, and
this lengthens the code and burdens someone reading the code with
remembering the invented names.  For example, Ben thinks the definition

\beginlisp
(define (rotate-around-origin theta)
  (let ((cth (cos theta))
        (sth (sin theta)))
    (lambda (curve)
      (lambda (t)
        (let ((ct (curve t)))      ;Ben eliminates the declaration of ct
          (let ((x (x-of ct))
                (y (y-of ct)))
            (make-point
             (- (* cth x) (* sth y))
             (+ (* sth x) (* cth y)))))))))
\endlisp

would be a bit more readable if the name {\tt ct} for the value of {\tt
(curve t)} was dropped.  He proposes instead:

\beginlisp
(define (bens-rotate theta)
  (let ((cth (cos theta))
        (sth (sin theta)))
    (lambda (curve)
      (lambda (t)
        (let ((x (x-of (curve t)))      ;Ben writes (curve t)
              (y (y-of (curve t))))     ;twice
          (make-point
           (- (* cth x) (* sth y))
           (+ (* sth x) (* cth y))))))))
\endlisp

Is Ben's definition correct?

Alyssa P. Hacker warns Ben that the {\tt let} declarations are more
significant computationally than mere abbreviations.  Briefly explain why
using {\tt bens-rotate} as a subprocedure in place of the original {\tt
rotate-around-origin} in the definition of {\tt gosper-curve} will turn a
process whose time is linear in the level into one which is exponential in
the level.

\paragraph{Exercise 8.B}
Look up the online documentation of the {\tt trace-entry} procedure in the
Scheme Users Manual.  Trace {\tt x-of} to show how dramatically Alyssa's
warning is confirmed when computing points on the gosper curve using {\tt
bens-rotate} as a subprocedure in place of the original {\tt
rotate-around-origin}.  Turn in a table summarizing the number of calls to
{\tt x-of} by {\tt gosper-curve} using the two different rotating
procedures at four or five illustrative levels.  (A simple way to switch
to use of {\tt bens-rotate} in place of {\tt rotate-around-origin} is to
evaluate

\beginlisp
(define rotate-around-origin bens-rotate)
\endlisp
Of course, you had better save the procedure {\tt rotate-around-origin}
under some other name so you can restore it.  Otherwise, you may have to
reload the problem set.)

\medskip

We can now invent other schemes like the Gosper process, and use them
to generate fractal curves.

\paragraph{Exercise 9.A}  The {\em Koch curve} is produced by a
process similar to the Gosper curve, as shown in figure~\ref{Koch}.
Write a procedure {\tt kochize} that generates Koch curves.
% \label{Koch} here

\begin{figure}
\picture 5.15 in by 1.85 in (Koch)
\caption{{\protect\footnotesize
The Koch at various levels, and a ``snowflake'' curve formed from
three Koch curves.  As with the C curve, each level approximation
to the Koch curve is obtained by
applying the a transformation to the previous level.}}
\label{Koch}
\end{figure}

(Teaser: You can generate the Koch curve by using {\tt param-gosper} with
an appropriate argument.  Can you find this?)


\paragraph{Exercise 9.B} Print some pictures of your Koch curve at
various levels.

\paragraph{Exercise 10 (Optional)}

You now have a lot of elements to work with: scaling, rotation,
translation, curve plotting, Gosper processes, Koch processes, and
generalizations.  For example, you can easily generalize the
parameterized Gosper process to start with something other than an
horizontal line.  The Gosper curve is continuous but {\em nowhere
differentiable}, so it may be interesting to display its derivatives
at various levels and numbers of points (see the procedure {\tt
deriv-t} in the file {\tt curves.scm}).  Or you can create new fractal
processes.  Or you can combine the results of different processes into
one picture.  Spend some time playing with these ideas to see what you
can come up with.

\end{document}

